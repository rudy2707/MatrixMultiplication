\documentclass[a4paper]{article}

% Packages
\usepackage{listings}
\usepackage{color}
\usepackage[utf8]{inputenc}
\usepackage{listingsutf8}
\usepackage{graphicx}
\usepackage{epstopdf}
\usepackage{fancyhdr}
\usepackage[T1]{fontenc}
%\usepackage[top=2cm, bottom=2cm, left=3.5cm, right=2cm]{geometry} % Les marges.
\usepackage[top=2cm, bottom=2cm, left=2cm, right=2cm]{geometry} % Les marges.

\definecolor{mygreen}{rgb}{0,0.6,0}
\definecolor{mygray}{rgb}{0.5,0.5,0.5}
\definecolor{mymauve}{rgb}{0.58,0,0.82}
\definecolor{bggray}{rgb}{0.95, 0.95, 0.95}
\lstset{inputencoding=utf8/latin1}
\lstset{ %
    backgroundcolor=\color{bggray},   % choose the background color; you must add \usepackage{color} or \usepackage{xcolor}
    basicstyle=\footnotesize,        % the size of the fonts that are used for the code
    breakatwhitespace=false,         % sets if automatic breaks should only happen at whitespace
    breaklines=true,                 % sets automatic line breaking
    captionpos=b,                    % sets the caption-position to bottom
    commentstyle=\color{mygreen},    % comment style
    deletekeywords={...},            % if you want to delete keywords from the given language
    escapeinside={\%*}{*)},          % if you want to add LaTeX within your code
    extendedchars=true,              % lets you use non-ASCII characters; for 8-bits encodings only, does not work with UTF-8
    frame=single,                    % adds a frame around the code
    frameround=tttt                  % tttt for having the corner round.
    keepspaces=true,                 % keeps spaces in text, useful for keeping indentation of code (possibly needs columns=flexible)
    keywordstyle=\color{blue},       % keyword style
    language=Matlab,                 % the language of the code
    morekeywords={*,...},            % if you want to add more keywords to the set
    numbers=left,                    % where to put the line-numbers; possible values are (none, left, right)
    numbersep=5pt,                   % how far the line-numbers are from the code
    numberstyle=\tiny\color{mygray}, % the style that is used for the line-numbers
    rulecolor=\color{black},         % if not set, the frame-color may be changed on line-breaks within not-black text (e.g. comments (green here))
    showspaces=false,                % show spaces everywhere adding particular underscores; it overrides 'showstringspaces'
    showstringspaces=false,          % underline spaces within strings only
    showtabs=false,                  % show tabs within strings adding particular underscores
    stepnumber=1,                    % the step between two line-numbers. If it's 1, each line will be numbered
    stringstyle=\color{mymauve},     % string literal style
    tabsize=2,                       % sets default tabsize to 2 spaces
    title=\lstname                   % show the filename of files included with \lstinputlisting; also try caption instead of title
}

% Header
\pagestyle{fancy}
\fancyhead[L]{Axel Fahy \& Jessy Marin \& Rudolf Höhn}
\fancyhead[R]{\today}

\title{TP Matrix multiplication\\Analyse de performance}
\author{Axel Fahy \& Jessy Marin \& Rudolf Höhn}
\date{\today}

\begin{document}
\maketitle

\section{Fox}

\subsection{Complexité}
Pour trouver la complexité totale, on doit séparer le problème. On doit trouver la complexité pour le broadcast de T, la création de la matrice S et de la multiplicaiton matricielle.
Dans les formules, $\beta = T_{bande passante}, \alpha = T_{execution}$.
\begin{itemize}
    \item \textbf{Nombre d'étapes} = $\sqrt[2]{p}$ (p = nombre de processeurs)
    \item \textbf{\boldmath$T_{broadcast}$} = $\beta \frac{n^2}{p} log_2(\sqrt[2]{p}) = \beta \frac{n^2}{p} \frac{log_2(p)}{2}$
    \item \textbf{\boldmath$T_{shift}$} = $\beta \frac{n^2}{p}$
    \item \textbf{\boldmath$T_{communication}$} = $T_{broadcast} + T_{shift}$
    \item \textbf{\boldmath$T_{multiplication}$}\\
    Il y a $\sqrt[2]{p}$ étapes et donc le même nombre de multiplication de sous-matrices.\\
    La taille des sous-matrices à multiplier est de $(\frac{n}{\sqrt[2]{p}} \cdot \frac{n}{\sqrt[2]{p}})$\\
    La complexité du calcul est donc : $\alpha \sqrt[2]{p} \cdot (\frac{n}{\sqrt[2]{p}})^3 = \alpha \frac{n^3}{p}$
\item \textbf{\boldmath$T_{par}$} =
\end{itemize}

La complexité totale est donc :
\begin{center}
\textbf{\boldmath$T_{par}$} = $T_{communication} + T_{calc} = \frac{\alpha n^3}{p} + \beta \frac{n^2}{\sqrt[2]{p}}(\frac{log_2(p)}{2} + 1)$
\end{center}

\subsection{Speedup}
Le speedup théorique est : $S = \frac{T_{seq}}{T_{par}}$. Dans notre cas, le speedup est :
\begin{center}
    $S=\frac{n^3}{\frac{\alpha n^3}{p} + \beta \frac{n^2}{\sqrt[2]{p}}(\frac{log_2(p)}{2} + 1)}$
\end{center}
\subsection{Efficacité}
$E(p) = \frac{S(p)}{p} = \frac{\frac{n^3}{\frac{\alpha n^3}{p} + \beta \frac{n^2}{\sqrt[2]{p}}(\frac{log_2(p)}{2} + 1)}}{p}$
\subsection{Isoefficacité}
Pour atteindre une efficacité constante, il faut que n soit égal à $n = c\sqrt[2]{p} (\frac{log_2(p)}{2} + 1)$

\newpage

\section{Cannon}

\subsection{Complexité}
Pour calculer la complexité il faut d'abord calculer le poid de toutes les manipulations :\newline
($\beta = T_{bande passante}, \alpha = T_{execution}$, p = nombre de processeurs)
\begin{itemize}
    \item \textbf{Construction des matrices T(k) et S(k)}\newline
    $\sqrt[2]{p} \cdot \beta \cdot \frac{n^2}{p}$
    \item \textbf{Temps de communication}\newline
    $2 \cdot \sqrt[2]{p} \cdot \beta \cdot \frac{n^2}{p} = 2 \cdot \beta \cdot \frac{n^2}{\sqrt[2]{p}}$
    \item \textbf{Temps de Calcul}
    \begin{itemize}
    	\item \textbf{Addition :} $(\sqrt[2]{p}-1) \cdot \alpha \cdot (\frac{n}{\sqrt[2]{p}})^2 = (\sqrt[2]{p}-1) \cdot \alpha \cdot \frac{n^2}{p}$
    	\item \textbf{Multiplication :} $\sqrt[2]{p} \cdot \alpha (\frac{n}{\sqrt[2]{p}})^3 = \sqrt[2]{p} \cdot \alpha \cdot \frac{n^3}{p^2} = \alpha \cdot \frac{n^3}{p}$
    	\item \textbf{Total :} $\alpha \cdot \frac{n^3}{p}$
    	\color{red}\fbox{\color{black}$ + (\sqrt[2]{p}-1) \cdot  \alpha \cdot \frac{n^2}{p}$} \color{red}le temps de l'addition est négligeable
    \end{itemize}
\end{itemize}
Ensuite nous pouvons calculer le temps séquentiel et parallèle :
\begin{itemize}
\item{Séquentiel :}\newline$T_{seq}(n \times n) = \alpha \cdot n^3$\\
\item{Parallèle :}\newline$T_{par}(p,n) = T_{calcul}+T_{Communication} = \alpha \cdot \frac{n^3}{p} + 2 \cdot \beta \cdot \frac{n^2}{\sqrt[2]{p}}$
\end{itemize}

\subsection{Speedup}
Pour le Speedup nous reprenons la formule de base :
\begin{center}
$Speedup = \frac{T_{seq}}{T_{par}}$
\end{center}
Et ajoutons les calculs précédent calculé :
\begin{center}
$Speedup = \frac{\alpha \cdot n^3}{\alpha \cdot \frac{n^3}{p} + 2 \cdot \beta \cdot \frac{n^2}{\sqrt[2]{p}}} = \frac{p}{1+\frac{2 \cdot \beta}{\alpha} \cdot \frac{\sqrt[2]{p}}{n}}$
\end{center}

\subsection{Efficacité}
Reprise de la formule de base : 
\begin{center}
$E(p,n \times n) = \frac{T_{seq}}{p \cdot T_{par}}$
\end{center}
Ajout des calculs précédent :
\begin{center}
$E(p,n \times n) = \frac{\alpha \cdot n^3}{p \cdot (\alpha \cdot \frac{n^3}{p}+2 \cdot \beta \cdot \frac{n^2}{\sqrt[2]{p}})} = \frac{\alpha \cdot n^3}{\alpha \cdot n^3+2 \cdot \beta \cdot \sqrt[2]{p} \cdot n^2} = \frac{1}{1+\frac{2 \cdot \beta}{\alpha} \cdot \frac{\sqrt[2]{p}}n}$
\end{center}
\end{document}


